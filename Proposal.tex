\documentclass[12pt,twocolumn]{article}

% Packages
\usepackage{graphicx} % For including images
\usepackage{listings} % For including code snippets
\usepackage{xcolor} % For custom colors
\usepackage{lipsum} % For placeholder text
\usepackage{hyperref} % For hyperlinks
\usepackage{geometry} % For customizing page layout

% Define custom colors
\definecolor{codegreen}{rgb}{0,0.6,0}
\definecolor{codegray}{rgb}{0.5,0.5,0.5}
\definecolor{codepurple}{rgb}{0.58,0,0.82}
\definecolor{backcolour}{rgb}{0.95,0.95,0.92}

% Define code listing style
\lstdefinestyle{mystyle}{
    backgroundcolor=\color{backcolour},   
    commentstyle=\color{codegreen},
    keywordstyle=\color{magenta},
    numberstyle=\tiny\color{codegray},
    stringstyle=\color{codepurple},
    basicstyle=\ttfamily\footnotesize,
    breakatwhitespace=false,         
    breaklines=true,                 
    captionpos=b,                    
    keepspaces=true,                 
    numbers=left,                    
    numbersep=5pt,                  
    showspaces=false,                
    showstringspaces=false,
    showtabs=false,                  
    tabsize=2
}
\lstset{style=mystyle}

% Title and author information
\title{CS Proposal: Developing a Community-Centric Platform for Financial Literacy}
\author{Mattia Sparacino \\ \texttt{sparacino@oxy.edu} \\ Occidental College}
\date{Friday May 3rd 2024}

% Page layout
\geometry{margin=0.75in}

\begin{document}

\maketitle

\section{ Introduction and Problem Context}

In today's world, where financial literacy is important but often lacking, the need for accessible, clear, and reliable financial education is becoming more important as prices of living continuously increase. This project proposes the development of an application that helps individuals gain financial information and education through a community. Unlike traditional financial apps that focus on transactions and investments, this platform will concentrate on providing guides, tutorials, and interactive tools designed to help users understand personal finance management. The mission is to empower users by giving them the knowledge needed to make informed financial decisions.

Transitioning from the original idea of creating a financial app to developing an educational platform addresses several ethical concerns related to the app. Traditional finance apps often have issues related with data privacy, algorithmic bias, and accessibility problems, which can disadvantage certain user groups. By focusing on education, this app can help remove or diminish the ethical issues that would come with an actual financial app. With my idea and goal to emphasize transparency means that users understand not just what financial tools are available but also how to use and understand them, therefore, allowing them to reach they're personal goals. 

By designing an application that is accessible and diverse, I want to also ensure that this education reaches as broad an audience as possible. The goal is not only to create a wider understanding of financial concepts but also to build a community where users can share experiences and learn from each other. By doing so, the platform not only makes knowledge more digestible and interesting but also empowers users to contribute and shape the content, therefore, enhancing user engagement. This community approach in my opinion, consequently provides a strong foundation against the ethical issues around traditional financial applications, and allows for a more ethical and sustainable alternative in financial technology.



\section{Technical Background}
For the technology side of the platform, I am think of using web technologies and content management systems that are used for the creation, management, and distribution of the app's content. Primarily, the platform will use WordPress or a similar content management systems as it seems very flexible and seems to have a great support community. My goal is that the system will be made with plugins and custom modules to provide interactive elements such as quizzes and things of that nature which are essential for practical learning.

For interactive web technologies, the platform will use JavaScript along with other things like Vue.js. I want to also implement dynamic user interfaces that are both engaging and informative. These technologies allow the app to have personal interactions with users and personalized learning information, which can change based on the user’s progress and preferences.

Accessibility and usability are also very important, as they ensure that the platform is diverse and meets all user needs. I want to ensure that the platform is  usable for individuals with disabilities, including those who use screen readers, have limited mobility, or require other accommodations. However, I have not looked at how I can do this quite yet but will make sure it is included in my research.



My design will prioritize a mobile app, as I recognize that many users get most of their content through their mobile devices. I want to also make sure the application is responsive to ensure the platform operates smoothly and looks good on all devices, including desktops, tablets, and smartphones.

To refine my user experience, I am going to also do user testing with diverse groups. This feedback will be crucial for ongoing improvements. My testing will focus on the clarity of the content, the intuitiveness of the navigation, and the overall satisfaction of my users. This process will help me create a platform that is both effective and enjoyable for all users.

By using these technologies and making sure accessibility and usability is focused on a lot, the platform will aim to provide an easy and inclusive educational experience that empowers users from everywhere to enhance their financial literacy.

\section{Prior Work}

The landscape of financial education platforms are varied, having everything from simple informational websites to extensive interactive learning systems and community forums. Major contributors like Investopedia, Khan Academy, and NerdWallet offer a large amount of resources, primarily focusing on delivering content. However, these platforms often emphasize content delivery at the expense of interactive learning and community engagement.

While these sites offer valuable information, they can fall short in areas like user-centric design and data privacy. These aspects are important for creating a trustworthy educational environment. A focus on enhancing user experience and ensuring great privacy measures can make a significant difference in how these platforms serve their users and build trust.


Many already existing platforms focus on using user data primarily for  marketing, which often compromises user privacy and not necessarily align with learners' best interests. Additionally, while these platforms provide valuable information, they often lack mechanisms for user interaction, limiting people of having an actual understanding of the content. 

My platform aims to bridge these things by prioritizing ethical data practices and community involvement from the start. I am committed to making sure the data and information is minimal and transparent to ensure that any data collected is used solely to enhance experience. Users will be fully informed about how their data is used and how it's not being used for anything but benefiting them. 

Moreover, my platform will put a emphasis on community engagement. In contrast to traditional apps and models, it will feature interactive forums and collaborative projects that encourage users to discuss topics, share personal experiences, and learn from one another. By creating a supportive community, my platform aims to not only enhance the educational experience but also builds a network of learners who can continue to grow together



\section{Methods}

My approach to developing this platform is comprehensive, focusing on both content creation and community engagement to make sure I take into consideration the highest ethical standards and best practices in educational technology.

Content Creation and Features
\subsection{Unbiased Content}

I am committed to providing impartial and comprehensive educational materials. To achieve this, all content will undergo a review process by financial experts (if I can find a way to do that) to eliminate any potential biases that could influence learners' decision-making.

\subsection{Multilingual Support}

In recognition of my idea of a diverse user base, the platform will offer multilingual support. This includes integrating translation features and developing content in several major languages, ensuring accessibility for non-English speakers.

\subsection{Accessibility Features}

The platform will be incorporating features such as text-to-speech, high-contrast visual elements, and easy-to-navigate layouts. These features will make the platform accessible and user-friendly for people with various disabilities, promoting greater inclusivity.

\subsection{Interactive Forums}

I will include moderated forums where users can interact, ask questions, and share insights on financial topics. These forums will also allow Q&A sessions with financial experts, if I can manage to scale the application and community which would be great as I could offer real-time support and enhance the learning experience.

\subsection{User Feedback Mechanisms}

To ensure the platform evolves according to user needs and preferences, I will implement continuous feedback mechanisms, like surveys, suggestion boxes, and different user testings. This feedback will be crucial for ongoing improvements, helping me stay aligned with my user expectations and ethical standards.

\subsection{Collaborative Content Development}

I really want to encourage and implement user participation in the content creation process through guest posts and story sharing. This collaborative model allows for the content to continue to empower users by giving them a role in shaping the material on the app.

Therefore, by implementing these methods, my platform aims to create a dynamic, ethical environment that not only makes the information easier and more fun to learn but also hopefully help cultivate a sense of ownership and community among users. This approach ensures that the platform remains relevant, user-friendly, and committed to the ethical principles guiding its development.


\section{Evaluation Metrics}

To ensure my platform's success and positive impact, I will also use a variety of evaluation metrics focused on user engagement, content clarity, and regards to ethics. These metrics will help assess the effectiveness of the educational content and the integrity of the platform's framework.

\subsection{User Engagement}

Engagement Metrics: I will monitor the quantitative metrics such as user session duration, number of active users, page views per visit, and frequency of logins. These indicators will help me understand user engagement, and reveal how compelling and user-friendly the platform is, and highlighting areas for improvement.

Community Activity: The level of participation in community forums and interaction with dynamic content (e.g., quizzes, polls) will also be tracked and interaction will show an engaged community.

\subsection{Content Clarity}

User Surveys: I will have post-interaction surveys for each of the lessons and such that will collect user feedback on the clarity and utility of the content. Users will rate aspects like understandability, relevance, and satisfaction with how the content meets their educational needs.

Content Review Sessions: I wat to implement focus groups with users that will provide feedback on various content areas. These sessions will be critical for refining content to ensure it is clear and understandable across user groups.

\subsection{Ethical Adherence}

Accessibility Reviews: In the testing of my app I will do accessibility reviews and will include tests conducted by users with disabilities to collect authentic feedback on their user experience.

Ethical Feedback Mechanism: A  channel for reporting and discussing ethical concerns will be made on the app, ensuring transparency and ongoing conversation with users. This will cover issues like content bias, data misuse, or other ethical challenges.


By implementing these things, I aim to make a responsive, user-focused educational environment that upholds and reinforces ethical principles in financial education. The data collected and user testing will guide ongoing development, ensuring the platform remains effective and relevant for its users.


\section{Ethical Considerations}

The development and management of an educational platform dealing with sensitive personal information and focusing on financial education requires me to ensure that I include things that will ensure I am adhering to ethical standards. Therefore, this section outlines the essential ethical considerations and the  measures I will take to ensure these standards are  met.

\subsection{Accuracy of Information}

Content Verification Process: To ensure the accuracy and reliability of my content, all material will be fully researched and looked at on my end. I will be determining the truthfulness of the information I decide to incorporate on the app and ensure that everything is cross-referenced with current, credible financial sources. The app will also include frequent changes if any of the information is outdated and therefore I  will regularly update the content to reflect the most current information, ensuring its relevance and accuracy.

\subsection{Privacy of User Data}

Data Protection Measures: I will implement security measures, which I will look at more at a later date to ensure that users can protect their information if they so choose to do. I also want some type of authentication and verification as  want this to be a positive community and do not want "trolls" or people that do not belong using the app for unsanctioned topics or communication. 

Transparency and User Control: To ensure this is adhered to, the users will be fully informed about the data collected, its use, and sharing policies through transparent, easily accessible privacy policies. They will also have control over their data, with options to view, modify, and delete their information.

\subsection{Inclusivity of Content}

Diverse Representation: My content will be designed to be inclusive, targeting a diverse audience across various backgrounds, educational levels, and financial knowledge. I want to  ensure that the language is culturally unbiased and accessible to non-experts.

Accessibility Standards: I want to ensure that the app commits to adhering to Web Content Accessibility Guidelines to ensure that my platform is accessible to individuals with various disabilities. This includes providing alternative text for images, ensuring sufficient color contrast, and enabling text-to-speech functionality as mentioned before in the methods section. 

\subsection{Proactive Measures for Ethical Maintenance}

Feedback Mechanisms: I will ensure the application includes feedback mechanisms that will be made for users to express concerns in the community forums, and content, furthermore, the app will also have built in periodic surveys in order to ensure I am keeping up to date with user wants. 

By implementing these measures, my platform aims to uphold standards of ethics and change the lack of these things in traditional financial education. My commitment to ethics is crucial for building and maintaining trust with our users, which is vital for the success and sustainability of the platform.

\section{Timeline}

The development of my platform will be done in structured phases, each will have exact things that I want to do in that time frame to ensure that the creation, testing and refinement is done well before the actual release happens. This also ensures that the project is complete by the December 2024 deadline.

\subsection{Phase 1: Research and Planning (June 2024 - July 2024)}

June 2024: I will begin with a market analysis to understand potential user needs and identify gaps in existing financial education platforms. Furthermore, I will build a detailed analysis of all the information I want to include in my app and all of the information of what the app will look like.

July 2024: This month will be dedicated to developing and finalizing the project plan, which will outline technical specifications, ethical guidelines, and initial design examples. The goal of this is to hone in on all of the information I want included, what I visibly want the app to look like and what the process is for users to start the app. This will be extensve as I want to fully plan out how all the data and information will look like and how I will categorize all of the information and parts of my app. 

\subsection{Phase 2: Content Development and Platform Setup (August 2024 - October 2024)}

August 2024: Here, I will actually start the development of the app and start implementing the backend of the app. I will also look more into content development and will create my final focus on foundational financial topics. 

September 2024: Here, I will do as much of the backend of the code for the project and start on the front end. My goal is to finish the backend and towards the end of September, start implementing the front end ensuring that I have an idea of how the app will look and start developing this. 

October 2024: During this month, I want to see the completion of primary development for interactive elements and community features.  want to get a first beta done to see what the app looks like and how everything on t looks. If I am happy with this I will start user testing and will create as script ensuring the testers look into my accessibility features and ethical considerations. 

\subsection{Phase 3: Beta Testing and Refinement (November 2024 - December 2024)}

November 2024: A beta version of the app will be released to a select group of users after the first round, this time without looking for specific things but actually using the app and the functionality to collect extensive feedback on usability, content clarity, and ethical adherence.

December 2024: Feedback from the beta testing will be analyzed, and necessary adjustments will be made to the platform and content to finalize preparations for the final product.

\subsection{Phase 4: Official Launch and Post-Launch Activities (December 2024)}

Early December 2024: The platform will be launched to the public, if there is interest. I will do outreach as I am very committed to seeing if anyone would be interested in this application, and if not I am perfectly happy sharing it with family and friends. 

By following this detailed timeline, I aim to efficiently manage the project's progress and ensure a successful completion with attention to everything specified in this proposal.

\section{Conclusion}

Therefore, to conclude, this proposal has laid out a detailed plan for the development of a community based platform dedicated to enhancing financial literacy. By focusing on accessible, clear, and engaging educational content, this platform seeks to empower users with the knowledge and tools necessary to navigate their personal financial goals wthout having to trust someone or information that is not confirmed. The ethical framework built into the platform ensures that users can trust in the accuracy and privacy of the information provided.

From  content verification to inclusive design and user engagement mechanisms, the strategies outlined in this proposal aim to address the inaccessibility of existing financial education platforms. By creating an environment that fosters community interaction and continual learning, the platform will not only educate but also connect users, enriching their overall learning experience and therefore, benefiting for their future.


In conclusion, the project represents a step forward in educational technology, one that prioritizes user needs, ethical standards, and educational teaching. With a vision and a detailed road map, I believe this project could  make a significant impact on the way financial education is approached, making it more accessible, interactive, and empowering for all users. 

\end{document}
